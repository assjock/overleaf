
\documentclass{article}
\usepackage[centertags]{amsmath}
\usepackage{amsfonts}
\usepackage{amssymb}
\usepackage{amsthm}
\usepackage{newlfont}
\usepackage{setspace}
\newcommand{\co}{\mathrm{const}}
\newcommand{\e}{\epsilon}
\newcommand{\la}{\lambda}
\newcommand{\G}{\Gamma}
\begin{document}
\title{Trigonometry Problems}\author{Amir Hossein Parvardi\thanks{ Email: a.parvardi@gmail.com, website: parvardi.com/ }} \maketitle

\doublespacing

\noindent 
{\bf 1.} Prove that:

\[\cos\frac{2\pi}{13}+\cos\frac{6\pi}{13}+\cos\frac{8\pi}{13}=\frac{\sqrt{13}-1}{4}\]

\noindent
{\bf 2.} Prove that $2\left(\cos\frac{4\pi}{19}+\cos\frac{6\pi}{19}+\cos\frac{10\pi}{19}\right)$ is a root of the equation:
\[\sqrt{4+\sqrt{4+\sqrt{4-x}}}=x\]

\noindent
{\bf 3.} Prove that
\[\sqrt{\frac12+\frac12 \sqrt{\frac12+\frac12 \sqrt{\frac12+\frac12 \cos 8\theta}}}=\cos \theta\]

\noindent
{\bf 4.} Prove that
\[ \sin^4 \left(\frac{\pi}{8}\right)+\sin^4 \left(\frac{3\pi}{8}\right)+\sin^4 \left(\frac{5\pi}{8}\right)+\sin^4 \left(\frac{7\pi}{8}\right)=\frac32\]


\noindent
{\bf 5.} Prove that
\[\cos x \cdot \cos \left(\frac{x}{2}\right) \cdot \cos \left(\frac{x}{4}\right) \cdot \cos \left(\frac{x}{8}\right)= \dfrac{\sin 2x}{16 \sin \left(\frac{x}{8}\right)}\]

\singlespacing
\noindent
{\bf 6.} Prove that
\[64 \cdot \sin 10^{\circ}\cdot \sin 20^{\circ}\cdot \sin 30^{\circ}\cdot \sin 40^{\circ}\cdot \sin 50^{\circ}\cdot \sin 60^{\circ}\cdot \sin 70^{\circ}\cdot \sin 80^{\circ}\cdot \sin 90^{\circ}=\dfrac34\]

\doublespacing
\noindent
{\bf 7.} Find $x$ if
\[\sin x= \tan12^{\circ}\cdot \tan48^{\circ}\cdot \tan54^{\circ} \cdot \tan72^{\circ}\cdot\]

\noindent
{\bf 8.} Solve the following equations in $\mathbb R$:
\begin{itemize}

\item $\sin 9x+ \sin 5x + 2\sin^2 x=1$

\item $\cos 5x \cdot \cos 3x - \sin 3x \cdot \sin x = \cos 2x$

\item $\cos 5x+\cos 3x+\sin 5x+\sin 3x=2 \cdot \cos\left(\frac{\pi}4 - 4x\right)$

\item $\sin x + \cos x -\sin x \cdot \cos x= -1$

\item $\sin 2x - \sqrt 3 \cos 2x = 2$


\end{itemize}


\noindent
{\bf 9.} Prove following equations:

\begin{itemize}

\item $\sin \left(\frac{2\pi}{7}\right)+\sin \left(\frac{4\pi}{7}\right) - \sin \left(\frac{6\pi}{7}\right)= 4 \sin \left(\frac{\pi}{7}\right) \cdot \sin \left(\frac{3\pi}{7}\right) \cdot \sin \left(\frac{5\pi}{7}\right)$

\item $\cos \left(\frac{\pi}{13}\right)+\cos \left(\frac{3\pi}{13}\right)+\cos \left(\frac{5\pi}{13}\right)+\cos \left(\frac{7\pi}{13}\right)+\cos \left(\frac{9\pi}{13}\right)+\cos \left(\frac{11\pi}{13}\right)=\frac12$

\item $\forall k \in \mathbb N \: \: : \: \: \cos \left(\frac{\pi}{2k+1}\right)+\cos \left(\frac{3\pi}{2k+1}\right)+\cdots +\cos \left(\frac{(2k-1)\pi}{2k+1}\right)=\frac12$

\item $\sin \left(\frac{\pi}{7}\right)+\sin \left(\frac{2\pi}{7}\right)+\sin \left(\frac{3\pi}{7}\right)=\dfrac 14 \cdot \cot \left(\frac{\pi}{4}\right)$

\end{itemize}


\noindent
{\bf 10.} Show that
\[\cos \frac{\pi}{n}  + \cos \frac{2 \pi}{n}+ \cdots + \cos \frac{n \pi}{n} = -1.\]

\noindent
{\bf 11.}Show that $\cos a + \cos 3a + \cos 5a + \cdots + \cos(2n - 1)a = \frac{\sin 2na}{2 \sin a}.$


\noindent
{\bf 12.} Show that $\sin a + \sin 3a + \sin 5a + \cdots + \sin(2n- 1)a = \frac{\sin^2 na}{\sin a}.$

\noindent
{\bf 13.} Calculate
\[(\tan1^\circ)^2+(\tan2^\circ)^2+(\tan3^\circ)^2+\ldots+(\tan89^\circ)^2.\]

\noindent
{\bf 14.} Prove that $\cot^2 \frac{\pi}{7}+\cot^2 \frac{2\pi}{7}+\cot^2 \frac{3\pi}{7}=5$.

\noindent
{\bf 15.} Show that $\tan{\frac {\pi}7}\tan{\frac {2\pi}7}\tan{\frac {3\pi}7}=\sqrt 7$.

\singlespacing
\noindent
{\bf 16.} $\cos\left(\frac{2\pi}{7}\right)$, $\cos\left(\frac{4\pi}{7}\right)$ and $\cos\left(\frac{6\pi}{7}\right)$ are the roots of an equation of the form $ax^3+bx^2+cx+d = 0$ where $a, b, c, d$ are integers. Determine $a, b, c$ and $d$.

\doublespacing
\noindent
{\bf *17.} Find the value of the sum
\[\sqrt[3]{\cos{\frac{2\pi}{7}}} + \sqrt[3]{\cos\frac{4\pi}{7}} + \sqrt[3]{\cos\frac{6\pi}{7}}.\]

\noindent
{\bf 18.} Solve the equation \[2\sin^4x(\sin 2x-3)-2\sin^2x(\sin 2x-3)-1=0.\]

\noindent
{\bf 19.} Express the sum of the following series in terms of $\sin x$ and $\cos x.$
\[\sum_{k=0}^n (2k+1)\sin ^ 2 \left(x+\frac{k}{2}\pi\right)\]

\noindent
{\bf 20.} Find the smallest positive integer $N$ for which
\[\frac{1}{\sin 45^\circ \cdot  \sin 46^\circ} + \frac{1}{\sin 47^\circ \cdot  \sin 48^\circ} +\cdots +\frac{1}{\sin 133^\circ \cdot  \sin 134^\circ} = \frac{1}{\sin N^\circ}.\]

\noindent
{\bf 21.} Find the value of
\[\frac{\sin 40^\circ + \sin 80^\circ}{\sin 110^\circ}.\]

\singlespacing
\noindent
{\bf 22.} Evaluate the sum \[ S=\tan 1^\circ  \cdot \tan 2^\circ + \tan 2^\circ  \cdot \tan 3^\circ +\tan 3^\circ  \cdot \tan 4^\circ +\cdots + \tan 2004^\circ  \cdot \tan 2005^\circ .\]


\noindent
{\bf 23.} Solve the equation :
\[ \sqrt {3} \sin x ( \cos x- \sin x ) + (2- \sqrt {6} ) \cos x+ 2 \sin x+ \sqrt {3} - 2 \sqrt {2} = 0.\]

\doublespacing
\noindent
{\bf 24.} Let $ f(x)=\frac{1}{\sin\frac{\pi x}{7}}$. Prove that $ f(3)+f(2)=f(1).$

\noindent
{\bf 25.} Suppose that real numbers $ x,y,z$ satisfy
\[ \dfrac{\cos{x}+\cos{y}+\cos{z}}{\cos{(x+y+z)}}=\dfrac{\sin{x}+\sin{y}+\sin{z}}{\sin{(x+y+z)}}=p\]
Prove that \[ \cos{(x+y)}+\cos{(y+z)}+\cos{(x+z)}=p.\]

\singlespacing
\noindent
{\bf 26.} Solve for $ \theta, 0 \le \theta \le \frac{\pi}{2}$:
\[ \sin^5 \theta + \cos^5 \theta =1.\]

\doublespacing
\noindent
{\bf 27.} For $ x,y\in [0, \frac{\pi}{3}]$ prove that $ \cos x+\cos y\leq 1+\cos xy.$

\singlespacing
\noindent
{\bf 28.} Prove that among any four distinct numbers from the interval $ (0,\frac{\pi}{2})$ there are two, say $ x,y,$ such that:
\[8 \cos x \cos y \cos (x-y)+1>4(\cos ^2 x+\cos ^2 y).\]

\noindent
{\bf 29.} Let $ B= \frac{\pi}{7}.$ Prove that
\[ \tan B \cdot \tan 2B +  \tan 2B \cdot \tan 4B + \tan 4B \cdot \tan B=-7.\]

\doublespacing
\noindent
{\bf 30.} a) Calculate
\[ \frac {1}{{\cos \frac {{6\pi }}{{13}}}} - 4\cos \frac {{4\pi }}{{13}} - 4\cos \frac {{5\pi }}{{13}} = ?
\]
b) Prove that
\[ \tan\frac {\pi }{{13}} + 4\sin \frac {{4\pi }}{{13}} = \tan\frac {{3\pi }}{{13}} + 4\sin \frac {{3\pi }}{{13}}
\]
c) Prove that
\[ \tan\frac {{2\pi }}{{13}} + 4\sin \frac {{6\pi }}{{13}} = \tan\frac {{5\pi }}{{13}} + 4\sin \frac {{2\pi }}{{13}}
\]

\noindent
{\bf 31.} Prove that if $ \alpha$, $ \beta$ are angles of a triangle and $ \left(\cos^2\alpha+ \cos^2\beta\right)\left(1+ \tan\alpha \cdot \tan\beta\right)= 2$, then $ \alpha + \beta= 90^{\circ}.$

\singlespacing
\noindent
{\bf 32.} Let $a,b,c,d \in [-\frac{\pi}{2}, \frac{\pi}{2}]$ be real numbers such that
$\sin{a}+\sin{b}+\sin{c}+\sin{d}=1$ and $\cos{2a}+\cos{2b}+\cos{2c}+\cos{2d}\geq \frac{10}{3}$.
Prove that $a,b,c,d \in [0, \frac{\pi}{6}]$

\doublespacing
\noindent
{\bf 33.} Find all integers $m,n$ for which we have $\sin^{m}x+\cos^{n}x=1$, for all $x.$

\noindent
{\bf 34.} Prove that $\tan 55^{\circ} \cdot \tan 65^{\circ} \cdot \tan 75^{\circ}=\tan 85^{\circ}.$

\noindent
{\bf 35.} Prove that $\frac{4 \cos12^{\circ}+4 \cos 36^{\circ}+1}{\sqrt{3}}=\tan 78^{\circ}.$

\noindent
{\bf 36.} Prove that
\[ \sqrt{4+\sqrt{4+\sqrt{4-\sqrt{4+\sqrt{4+\sqrt{4-\cdots}}}}}}=2\left(\cos\frac{4\pi}{19}+\cos\frac{6\pi}{19}+\cos\frac{10\pi}{19}\right).\]
The signs: $ ++-++-++-++-\cdots$

\noindent
{\bf 37.} For reals $x,y$ Prove that $\cos{x}+\cos{y}+\sin{x}\sin{y}\leq 2.$


\noindent
{\bf 38.} Solve the equation in real numbers
\[\sqrt{7+2\sqrt{7-2\sqrt{7-2x}}}=x.\]

\singlespacing
\noindent
{\bf 39.} Let $A,B,C$ be three angles of triangle $ABC$. Prove that
\[(1-\cos A)(1-\cos B)(1-\cos C)\geq \cos A\cos B\cos C.\]

\noindent
{\bf 40.} Solve the equation \[ \sin^3 \left( x \right) - \cos^3 \left( x \right) = \sin^2 \left( x \right).\]

\doublespacing
\noindent
{\bf 41.} Find $S_n=\sum_{k=1}^n\sin^2 k\theta$ for $n\geqslant 1$

\noindent
{\bf 42.} Prove the following without using induction:
\[\cos x + \cos 2x + \cdots + \cos nx = \frac{\cos \frac{n+1}{2}x \cdot \sin \frac{n}{2}x}{\sin \frac{x}{2}}.\]

\singlespacing
\noindent
{\bf 43.} Evaluate:
\[ \sin \theta + \frac{1}{2}\cdot \sin 2 \theta + \frac{1}{2^2} \cdot \sin 3 \theta + \frac{1}{2^3} \cdot \sin 4\theta+\cdots \]

\doublespacing
\noindent
{\bf 44.} Compute \[\displaystyle \sum_{k=1}^{n-1}\csc^2 \left( \frac{k\pi}{n} \right) .\]

\noindent
{\bf 45.} Prove that

\begin{itemize}


\item $ \quad \tan\theta+\tan\left(\theta+\frac{\pi}{n}\right)+\tan\left(\theta+\frac{2\pi}{n}\right)+\cdots +\tan\left[\theta+\frac{(n-1)\pi}{n}\right] $ $=-n\cot\left(n\theta+\frac{n\pi}{2}\right).$

\item $ \quad \cot\theta+\cot\left(\theta+\frac{\pi}{n}\right)+\cot\left(\theta+\frac{2\pi}{n}\right)+\cdots +\cot\left[\theta+\frac{(n-1)\pi}{n}\right]=n\cot n\theta.$

\end{itemize}

\noindent
{\bf 46.} Calculate
\[\sum_{n=1}^\infty 2^{2n}\sin^4 \frac{a}{2^n}.\]

\noindent
{\bf 47.} Compute the following sum:
\[\tan 1^\circ + \tan 5^\circ +\tan 9^\circ + \cdots +\tan 177^\circ.\]

\noindent
{\bf 48.} Show that for any positive integer $n>1$,

\begin{itemize}

\item $\displaystyle \sum_{k=0}^{n-1}\cos\frac{2\pi k^{2}}{n}=\frac{\sqrt{n}}{2}(1+\cos\frac{n\pi}{2}+\sin\frac{n\pi}{2})$

\item $\displaystyle \sum_{k=0}^{n-1}\sin\frac{2\pi k^{2}}{n}=\frac{\sqrt{n}}{2}(1+\cos\frac{n\pi}{2}-\sin\frac{n\pi}{2})$

\end{itemize}

\noindent
{\bf 49.} Evaluate the product
\[ \prod_{k=1}^{n}tan\frac{k\pi}{2(n+1)}.\]

\noindent
{\bf 50.} Prove that, $\displaystyle \sum_{k=1}^{n}(-1)^{k-1}\cot\frac{(2k-1)\pi}{4n}=n$ for even $n.$

\noindent
{\bf 51.} Prove that $\displaystyle \sum_{k=1}^{n}\cot^{2}\left\{\frac{(2k-1)\pi}{2n}\right\}=n(2n-1)$.

\noindent
{\bf 52.} Prove that $\displaystyle \sum_{k=1}^{n}\cot^{4}\left(\frac{k\pi}{2n+1}\right)=\frac{n(2n-1)(4n^{2}+10n-9)}{45}$.

\singlespacing
\noindent
{\bf 53.} Let $x$ be a real number with $0<x<\pi$. Prove that, for all natural numbers $n$, the sum $\sin x + \frac{\sin 3x}{3} + \frac{\sin 5x}{5}+\cdots+ \frac{\sin (2n-1)x}{2n-1}$ is positive.

\end{document}

            