
\documentclass{article}
\usepackage[centertags]{amsmath}
\usepackage{amsfonts}
\usepackage{amssymb}
\usepackage{amsthm}
\usepackage{newlfont}
\usepackage{hyperref}
\newcommand{\co}{\mathrm{const}}
\newcommand{\e}{\epsilon}
\newcommand{\la}{\lambda}
%\newcommand{\G}{\Gamma}
\newcommand{\plus}{+}
\newcommand{\minus}{-}
\newcommand{\equal}{=}

\begin{document}
\title{Geometry Problems}\author{Amir Hossein Parvardi\thanks{ Email: a.parvardi@gmail.com, website: parvardi.com/ }} \maketitle

\par {\bf Note.} Most of problems have solutions. Just click on the number beside the problem to open its page and see the solution! Problems posted by different authors, but all of them are nice! Happy Problem Solving!

$$ $$

\href{http://www.artofproblemsolving.com/Forum/viewtopic.php?p=1899886#p1899886}{\bf 1.} Circles $W_1,W_2$ intersect at $P,K$. $XY$ is common tangent of two circles which is nearer to $P$ and $X$ is on $W_1$ and $Y$ is on $W_2$. $XP$ intersects $W_2$ for the second time in $C$ and $YP$ intersects $W_1$ in $B$. Let $A$ be intersection point of $BX$ and $CY$. Prove that if $Q$ is the second intersection point of circumcircles of $ABC$ and $AXY$
$$\angle QXA=\angle QKP$$

$$ $$

\href{http://www.artofproblemsolving.com/Forum/viewtopic.php?p=1899885#p1899885}{\bf 2.} Let $M$ be an arbitrary point on side $BC$ of triangle $ABC$. $W$ is a circle which is tangent to $AB$ and $BM$ at $T$ and $K$ and is tangent to circumcircle of $AMC$ at $P$. Prove that if $TK||AM$, circumcircles of $APT$ and $KPC$ are tangent together.

$$ $$

\href{http://www.artofproblemsolving.com/Forum/viewtopic.php?p=1871430#p1871430}{\bf 3.} Let $ABC$ an isosceles triangle and $BC>AB=AC$. $D,M$ are respectively midpoints of $BC, AB$. $X$ is a point such that $BX\perp AC$ and $XD||AB$. $BX$ and $AD$ meet at $H$. If $P$ is intersection point of $DX$ and circumcircle of $AHX$ (other than $X$), prove that tangent from $A$ to circumcircle of triangle $AMP$ is parallel to $BC$.

$$ $$

\href{http://www.artofproblemsolving.com/Forum/viewtopic.php?p=1871420#p1871420}{\bf 4.} Let $O$, $H$ be the circumcenter and the orthogonal center of triangle $\triangle ABC$, respectively. Let $M$ and $N$ be the midpoints of $BH$ and $CH$. Define $B'$ on the circumcenter of $\triangle ABC$, such that $B$ and $B'$ are diametrically opposed. If $HONM$ is a cyclic quadrilateral, prove that $\overline{B'N}=\frac12\overline{AC}$.

$$ $$

$$ $$

\href{http://www.artofproblemsolving.com/Forum/viewtopic.php?p=1385942#p1385942}{\bf 5.} $ OX,OY$ are perpendicular. Assume that on $ OX$  we have wo fixed points $ P,P'$ on the same side of $ O$. $ I$ is a variable point that $ IP\equal{}IP'$. $ PI,P'I$ intersect $ OY$ at $ A,A'$.

\par a) If $ C,C'$ Prove that $ I,A,A',M$ are on a circle which is tangent to a fixed line and is tangent to a fixed circle.


\par b) Prove that $ IM$ passes through a fixed point.

$$ $$

\href{http://www.artofproblemsolving.com/Forum/viewtopic.php?p=1385932#p1385932}{\bf 6.} Let $ A,B,C,Q$ be fixed points on plane. $ M,N,P$ are intersection points of $ AQ,BQ,CQ$ with $ BC,CA,AB$. $ D',E',F'$ are tangency points of incircle of $ ABC$ with $ BC,CA,AB$. Tangents drawn from $ M,N,P$ (not triangle sides) to incircle of $ ABC$ make triangle $ DEF$. Prove that $ DD',EE',FF'$ intersect at $ Q$.

$$ $$

\href{http://www.artofproblemsolving.com/Forum/viewtopic.php?p=1385593#p1385593}{\bf 7.} Let $ ABC$ be a triangle. $ W_a$ is a circle with center on $ BC$ passing through $ A$ and perpendicular to circumcircle of $ ABC$. $ W_b,W_c$ are defined similarly. Prove that center of $ W_a,W_b,W_c$ are collinear.

$$ $$

\href{http://www.artofproblemsolving.com/Forum/viewtopic.php?p=1385555#p1385555}{\bf 8.} In tetrahedron $ABCD$, radius four circumcircles of four faces are equal. Prove that $AB=CD, AC=BD$ and $AD=BC.$

$$ $$

\href{http://www.artofproblemsolving.com/Forum/viewtopic.php?p=1385585#p1385585}{\bf 9.} Suppose that $ M$ is an arbitrary point on side $ BC$ of triangle $ ABC$. $ B_1,C_1$ are points on $ AB,AC$ such that $ MB = MB_1$ and $ MC = MC_1$.  Suppose that $ H,I$ are orthocenter of triangle $ ABC$ and incenter of triangle $ MB_1C_1$. Prove that $ A,B_1,H,I,C_1$ lie on a circle.

$$ $$

\href{http://www.artofproblemsolving.com/Forum/viewtopic.php?p=1370723#p1370723}{\bf 10.} Incircle of triangle $ ABC$ touches $ AB,AC$ at $ P,Q$. $ BI, CI$ intersect with $ PQ$ at $ K,L$. Prove that circumcircle of $ ILK$ is tangent to incircle of $ ABC$ if and only if $ AB+AC=3BC$.

$$ $$

\href{http://www.artofproblemsolving.com/Forum/viewtopic.php?p=124397#p124397}{\bf 11.} Let $ M$ and $ N$ be two points inside triangle $ ABC$ such that
$$ \angle MAB = \angle NAC\quad \mbox{and}\quad \angle MBA = \angle NBC.$$
Prove that
$$ \frac {AM \cdot AN}{AB \cdot AC} + \frac {BM \cdot BN}{BA \cdot BC} + \frac {CM \cdot CN}{CA \cdot CB} = 1.$$

\href{http://www.artofproblemsolving.com/Forum/viewtopic.php?p=191489#p191489}{\bf 12.} Let $ABCD$ be an arbitrary quadrilateral. The bisectors of external angles $A$ and $C$ of the quadrilateral intersect at $P$; the bisectors of external angles $B$ and $D$ intersect at $Q$. The lines $AB$ and $CD$ intersect at $E$, and the lines $BC$ and $DA$ intersect at $F$. Now we have two new angles: $E$ (this is the angle $\angle{AED}$) and $F$ (this is the angle $\angle{BFA}$). We also consider a point $R$ of intersection of the external bisectors of these angles. Prove that the points $P$, $Q$ and $R$ are collinear.

$$ $$

\href{http://www.artofproblemsolving.com/Forum/viewtopic.php?p=916020#p916020}{\bf 13.} Let $ ABC$ be a triangle. Squares $ AB_{c}B_{a}C$, $ CA_{b}A_{c}B$ and $ BC_{a}C_{b}A$ are outside the triangle. Square $ B_{c}B_{c}'B_{a}'B_{a}$ with center $ P$ is outside square $ AB_{c}B_{a}C$. Prove that $ BP,C_{a}B_{a}$ and $ A_{c}B_{c}$ are concurrent.

$$ $$

\href{http://www.artofproblemsolving.com/Forum/viewtopic.php?p=852412#p852412}{\bf 14.} Triangle $ABC$ is isosceles ($AB=AC$). From $A$, we draw a line $\ell$ parallel to $BC$. $P,Q$ are on perpendicular bisectors of $AB,AC$ such that $PQ\perp BC$. $M,N$ are points on $\ell$ such that angles $\angle APM$ and $\angle AQN$ are $\frac\pi2$. Prove that
$$\frac{1}{AM}+\frac1{AN}\leq\frac2{AB}$$


$$ $$

\href{http://www.artofproblemsolving.com/Forum/viewtopic.php?p=835050#p835050}{\bf 15.} In triangle $ABC$, $M$ is midpoint of $AC$, and $D$ is a point on $BC$ such that $DB=DM$. We know that $2BC^{2}-AC^{2}=AB.AC$. Prove that \[BD.DC=\frac{AC^{2}.AB}{2(AB+AC)}\]

$$ $$

\href{http://www.artofproblemsolving.com/Forum/viewtopic.php?p=641483#p641483}{\bf 16.} $H,I,O,N$ are orthogonal center, incenter, circumcenter, and Nagelian point of triangle $ABC$. $I_{a},I_{b},I_{c}$ are excenters of $ABC$ corresponding vertices $A,B,C$. $S$ is point that $O$ is midpoint of $HS$. Prove that centroid of triangles $I_{a}I_{b}I_{c}$ and $SIN$ concide.

$$ $$

\href{http://www.artofproblemsolving.com/Forum/viewtopic.php?p=638185#p638185}{\bf 17.} $ABCD$ is a convex quadrilateral. We draw its diagnals to divide the quadrilateral to four triabgles. $P$ is the intersection of diagnals. $I_{1},I_{2},I_{3},I_{4}$ are excenters of $PAD,PAB,PBC,PCD$(excenters corresponding vertex $P$). Prove that $I_{1},I_{2},I_{3},I_{4}$ lie on a circle iff $ABCD$ is a tangential quadrilateral.

$$ $$

\href{http://www.artofproblemsolving.com/Forum/viewtopic.php?p=634198#p634198}{\bf 18.} In triangle $ABC$, if $L,M,N$ are midpoints of $AB,AC,BC$. And $H$ is orthogonal center of triangle $ABC$, then prove that $$LH^{2}+MH^{2}+NH^{2}\leq \frac{1}{4}(AB^{2}+AC^{2}+BC^{2})$$

$$ $$

\href{http://www.artofproblemsolving.com/Forum/viewtopic.php?p=118673#p118673}{\bf 19.} Circles $S_1$ and $S_2$ intersect at points $P$ and $Q$. Distinct points $A_1$ and $B_1$ (not at $P$ or $Q$) are selected on $S_1$. The lines $A_1P$ and $B_1P$ meet $S_2$ again at $A_2$ and $B_2$ respectively, and the lines $A_1B_1$ and $A_2B_2$ meet at $C$.  Prove that, as $A_1$ and $B_1$ vary, the circumcentres of triangles $A_1A_2C$ all lie on one fixed circle.

$$ $$

\href{http://www.artofproblemsolving.com/Forum/viewtopic.php?p=118667#p118667}{\bf 20.} Let $B$ be a point on a circle $S_1$, and let $A$ be a point distinct from $B$ on the tangent at $B$ to $S_1$. Let $C$ be a point not on $S_1$ such that the line segment $AC$ meets $S_1$ at two distinct points. Let $S_2$ be the circle touching $AC$ at $C$ and touching $S_1$ at a point $D$ on the opposite side of $AC$ from $B$.  Prove that the circumcentre of triangle $BCD$ lies on the circumcircle of triangle $ABC$.

$$ $$

\href{http://www.artofproblemsolving.com/Forum/viewtopic.php?p=2126712#p2126712}{\bf 21.} The bisectors of the angles $A$ and $B$ of the triangle $ABC$ meet the sides $BC$ and $CA$ at the points $D$ and $E$, respectively. Assuming that $AE+BD=AB$, determine the angle $C$.

$$ $$

\href{http://www.artofproblemsolving.com/Forum/viewtopic.php?p=2117909#p2117909}{\bf 22.} Let $A$, $B$, $C$, $P$, $Q$, and $R$ be six concyclic points.  Show that if the Simson lines of $P$, $Q$, and $R$ with respect to triangle $ABC$ are concurrent, then the Simson lines of $A$, $B$, and $C$ with respect to triangle $PQR$ are concurrent.  Furthermore, show that the points of concurrence are the same.

$$ $$

\href{http://www.artofproblemsolving.com/Forum/viewtopic.php?p=2114602#p2114602}{\bf 23.} $ABC$ is a triangle, and $E$ and $F$ are points on the segments $BC$ and $CA$ respectively, such that $\frac{CE}{CB}+\frac{CF}{CA}=1$ and $\angle CEF=\angle CAB$. Suppose that $M$ is the midpoint of $EF$ and $G$ is the point of intersection between $CM$ and $AB$. Prove that triangle $FEG$ is similar to triangle $ABC$.

$$ $$

\href{http://www.artofproblemsolving.com/Forum/viewtopic.php?p=358032#p358032}{\bf 24.} Let $ABC$ be a triangle with $\angle C = 90^\circ$ and $CA \neq CB$. Let $CH$ be an altitude and $CL$ be an interior angle bisector. Show that for $X \neq C$ on the line $CL$, we have $\angle XAC \neq \angle XBC$. Also show that for $Y \neq C$ on the line $CH$ we have $\angle YAC \neq \angle YBC$.

$$ $$

\href{http://www.artofproblemsolving.com/Forum/viewtopic.php?p=113332#p113332}{\bf 25.} Given four points $A,$ $B,$ $C,$ $D$ on a circle such that $AB$ is a diameter and $CD$ is not a diameter. Show that the line joining the point of intersection of the tangents to the circle at the points $C$ and $D$ with the point of intersection of the lines $AC$ and $BD$ is perpendicular to the line $AB.$

$$ $$

\href{http://www.artofproblemsolving.com/Forum/viewtopic.php?p=1992097#p1992097
 }{\bf 27.} Given a triangle $ABC$ and $D$ be point on side $AC$ such that $AB=DC$ , $\angle BAC=60-2X$ , $\angle DBC=5X$ and $\angle BCA=3X$
prove that $X=10.$


$$ $$

\href{http://www.artofproblemsolving.com/Forum/viewtopic.php?p=333377#p333377
 }{\bf 28.} Prove that in any triangle $ABC$,
$$0 < \cot { \left( \frac{A}{4} \right)} - \tan{ \left( \frac{B}{4} \right) } - \tan{ \left( \frac{C}{4} \right) } - 1 < 2 \cot { \left( \frac{A}{2} \right) }.  $$


$$ $$


\href{http://www.artofproblemsolving.com/Forum/viewtopic.php?p=1099544#p1099544
 }{\bf 29.} Triangle $ \triangle ABC$ is given. Points $ D$ i $ E$ are on line  $ AB$ such that  $ D - A - B - E, AD = AC$ and $ BE = BC$. Bisector of internal angles at $ A$ and $ B$ intersect $ BC,AC$ at $ P$ and $ Q$, and circumcircle of  $ ABC$ at  $ M$ and  $ N$. Line which connects  $ A$ with center of circumcircle of $ BME$ and line which connects $ B$ and center of circumcircle of $ AND$ intersect at $ X$. Prove that $ CX \perp PQ$.



$$ $$

\href{http://www.artofproblemsolving.com/Forum/viewtopic.php?p=1004026#p1004026
 }{\bf 30.} Consider a circle with center $O$ and points $A,B$ on it such that $AB$ is not a diameter. Let $C$ be on the circle so that $AC$ bisects $OB.$ Let $AB$ and $OC$ intersect at $D$, $BC$ and $AO$ intersect at $F.$ Prove that $AF=CD.$




$$ $$

\href{ http://www.artofproblemsolving.com/Forum/viewtopic.php?p=962592#p962592
}{\bf 31.}  Let $ ABC$ be a triangle.$ X;Y$ are two points on $ AC;AB$,respectively.$ CY$ cuts $ BX$ at $ Z$ and $ AZ$ cut $ XY$ at $ H \ (AZ \perp XY)$. $ BHXC$ is a quadrilateral inscribed in a circle.
Prove that $ XB=XC.$



$$ $$

\href{ http://www.artofproblemsolving.com/Forum/viewtopic.php?p=22914#p22914
}{\bf 32.} Let $ABCD$ be a cyclic quadrilatedral, and let $L$ and $N$ be the midpoints of its diagonals $AC$ and $BD$, respectively. Suppose that the line $BD$ bisects the angle $ANC$. Prove that the line $AC$ bisects the angle $BLD$.



$$ $$

\href{http://www.artofproblemsolving.com/Forum/viewtopic.php?p=868538#p868538
 }{\bf 33.} A triangle $\triangle ABC$ is given, and let the external angle bisector of the angle $\angle A$ intersect the lines perpendicular to $BC$ and passing through $B$ and $C$ at the points $D$ and $E$, respectively. Prove that the line segments $BE$, $CD$, $AO$ are concurrent, where $O$ is the circumcenter of $\triangle ABC$.



$$ $$

\href{http://www.artofproblemsolving.com/Forum/viewtopic.php?p=727178#p727178}{\bf 34.} Let $ABCD$ be a convex quadrilateral. Denote $O\in AC\cap BD$. Ascertain and construct the positions of the points $M\in (AB)$ and $N\in (CD)$, $O\in MN$ so that the sum $\frac{MB}{MA}+\frac{NC}{ND}$ is minimum.



$$ $$

\href{ http://www.artofproblemsolving.com/Forum/viewtopic.php?p=365242#p365242}{\bf 35.} Let $ABC$ be a triangle, the middlepoints $M,N,P$ of the segments $[BC],[CA],[AM]$ respectively, the intersection $E\in AC\cap BP$ and the projection $R$ of the point $A$ on the line $MN$. Prove that $\widehat {ERN}\equiv \widehat {CRN}$.



$$ $$

\href{http://www.artofproblemsolving.com/Forum/viewtopic.php?p=659219#p659219
 }{\bf 36.} Two circles intersect at two points, one of them $X.$ Find Y on one circle and $Z$ on the other, so that $X, Y$ and $Z$ are collinear and $XY \cdot XZ$ is as large as possible.



$$ $$

\href{http://www.artofproblemsolving.com/Forum/viewtopic.php?p=337819#p337819
 }{\bf 37.} The points $A, B, C, D$ lie in this order on a circle $o$.  The point $S$ lies inside $o$ and has properties $\angle SAD=\angle SCB$ and $\angle SDA= \angle SBC$. Line which in which angle bisector of $\angle ASB$ in included cut the circle in points  $P$ and $Q$. Prove that $PS =QS$.



$$ $$

\href{ http://www.artofproblemsolving.com/Forum/viewtopic.php?p=246214#p246214
}{\bf 38.} Given a triangle $ ABC$. Let $ G$, $ I$, $ H$ be the centroid, the incenter and the orthocenter of triangle $ ABC$, respectively. Prove that $ \angle GIH > 90^{\circ}$.


$$ $$


\href{http://www.artofproblemsolving.com/Forum/viewtopic.php?p=342752#p342752
 }{\bf 39.} Let be given two parallel lines $k$ and $l$, and a circle not intersecting $k$. Consider a variable point $A$ on the line $k$. The two tangents from this point $A$ to the circle intersect the line $l$ at $B$ and $C$. Let $m$ be the line through the point $A$ and the midpoint of the segment $BC$. Prove that all the lines $m$ (as $A$ varies) have a common point.

$$ $$



\href{http://www.artofproblemsolving.com/Forum/viewtopic.php?p=198163#p198163
 }{\bf 40.} Let $ABCD$ be a convex quadrilateral with $AD\not\parallel BC$. Define the points $E=AD \cap BC$ and $I = AC\cap BD$. Prove that the triangles $EDC$ and $IAB$ have the same centroid if and only if $AB \parallel CD$ and $IC^{2}= IA \cdot AC$.


$$ $$


\href{http://www.artofproblemsolving.com/Forum/viewtopic.php?p=578901#p578901
 }{\bf 41.} Let $ABCD$ be a square. Denote the intersection $O\in AC\cap BD$. Exists a positive number $k$ so that for any point $M\in [OC]$ there is a point $N\in [OD]$ so that $AM\cdot BN=k^{2}$. Ascertain the geometrical locus of the intersection $L\in AN\cap BM\ .$


$$ $$


\href{ http://www.artofproblemsolving.com/Forum/viewtopic.php?p=447144#p447144
}{\bf 42.} Consider a right-angled triangle $ABC$ with the hypothenuse $AB=1$. The bisector of $\angle{ACB}$ cuts the medians $BE$ and $AF$ at $P$ and $M$, respectively. If ${AF}\cap{BE}=\{P\}$, determine the maximum value of the area of $\triangle{MNP}$.



$$ $$

\href{http://www.artofproblemsolving.com/Forum/viewtopic.php?p=445337#p445337
 }{\bf 43.} Let triangle $ABC$ be an isosceles triangle with $AB = AC$. Suppose that the angle bisector of its angle $\angle B$ meets the side $AC$ at a point $D$ and that $BC = BD+AD$.
Determine $\angle A$.



$$ $$

\href{http://www.artofproblemsolving.com/Forum/viewtopic.php?p=443384#p443384
 }{\bf 44.} Given a triangle with the area $S$, and let $a$, $b$, $c$ be the sidelengths of the triangle. Prove that $a^{2}+4b^{2}+12c^{2}\geq 32\cdot S$.


$$ $$


\href{http://www.artofproblemsolving.com/Forum/viewtopic.php?p=431965#p431965
 }{\bf 45.} In a right triangle  $ABC$ with $\angle A = 90$  we draw the bisector $AD$ .  Let    $DK \perp AC , DL \perp AB$ .  Lines $BK , CL$ meet each other at  point $H$ .  Prove that $AH \perp BC$.

$$ $$



\href{ http://www.artofproblemsolving.com/Forum/viewtopic.php?p=181685#p181685
}{\bf 46.} Let $H$ be the orthocenter of the acute triangle $ABC$. Let $BB'$ and $CC'$ be altitudes of the triangle ($B^‘ \in AC$, $C^‘ \in AB$). A variable line $\ell$ passing through $H$ intersects the segments $[BC']$ and $[CB']$ in $M$ and $N$. The perpendicular lines of $\ell$ from $M$ and $N$ intersect $BB'$ and $CC'$ in $P$ and $Q$. Determine the locus of the midpoint of the segment $[ PQ]$.


$$ $$


\href{ http://www.artofproblemsolving.com/Forum/viewtopic.php?p=584865#p584865
}{\bf 47.} Let  $ABC$ be a triangle whit $\ AH\bot\ BC$ and $\ BE$ the interior bisector of the angle$\ ABC$.If $\ m(\angle BEA)=45$, find $\ m(\angle EHC).$


$$ $$


\href{ http://www.artofproblemsolving.com/Forum/viewtopic.php?p=519896#p519896
}{\bf 48.} Let $\triangle ABC$ be an acute-angled triangle with $AB \not= AC$. Let $H$ be the orthocenter of triangle $ABC$, and let $M$ be the midpoint of the side $BC$. Let $D$ be a point on the side $AB$ and $E$ a point on the side $AC$ such that $AE=AD$ and the points $D$, $H$, $E$ are on the same line. Prove that the line $HM$ is perpendicular to the common chord of the circumscribed circles of triangle $\triangle ABC$ and triangle $\triangle ADE$.


$$ $$


\href{ http://www.artofproblemsolving.com/Forum/viewtopic.php?p=519895#p519895
}{\bf 49.} Let $D$ be inside the $\triangle ABC$ and $E$ on $AD$ different of $D$. Let $\omega_1$ and $\omega_2$ be the circumscribed circles of $\triangle BDE$ resp. $\triangle CDE$. $\omega_1$ and $\omega_2$ intersect $BC$ in the interior points $F$ resp. $G$. Let $X$ be the intersection between $DG$ and $AB$ and $Y$ the intersection between $DF$ and $AC$. Show that $XY$ is $\|$ to $BC$.


$$ $$


\href{ http://www.artofproblemsolving.com/Forum/viewtopic.php?p=493656#p493656
}{\bf 50.} Let $\triangle{ABC}$ be a triangle, $D$ the midpoint of $BC$, and $M$ be the midpoint of $AD$. The line $BM$ intersects the side $AC$ on the point $N$. Show that $AB$ is tangent to the circuncircle to the triangle $\triangle{NBC}$ if and only if the following equality is true:
$$\frac{{BM}}{{MN}} =\frac{({BC})^2}{({BN})^2}.$$





$$ $$

\href{ http://www.artofproblemsolving.com/Forum/viewtopic.php?p=480525#p480525
}{\bf 51.} Let $\triangle ABC$ be a traingle with sides a, b, c, and area K. Prove that
$$ 27 (b^2 + c^2 - a^2)^2 (c^2 + a^2 - b^2)^2 (a^2 + b^2 - c^2)^2 \le (4K)^6$$



$$ $$

\href{ http://www.artofproblemsolving.com/Forum/viewtopic.php?p=439274#p439274
}{\bf 52.} Given a triangle $ABC$ satisfying $AC+BC=3\cdot AB$. The incircle of triangle $ABC$ has center $I$ and touches the sides $BC$ and $CA$ at the points $D$ and $E$, respectively. Let $K$ and $L$ be the reflections of the points $D$ and $E$ with respect to $I$. Prove that the points $A$, $B$, $K$, $L$ lie on one circle.


$$ $$


\href{http://www.artofproblemsolving.com/Forum/viewtopic.php?p=476117#p476117
 }{\bf 53.} In an acute-angled triangle $ABC$, we are given that $2\cdot AB = AC + BC$. Show that the incenter of triangle $ABC$, the circumcenter of triangle $ABC$, the midpoint of $AC$ and the midpoint of $BC$ are concyclic.

$$ $$


\href{http://www.artofproblemsolving.com/Forum/viewtopic.php?p=463068#p463068
 }{\bf 54.} Let $ABC$ be a triangle, and $M$ the midpoint of its side $BC$. Let $\gamma$ be the incircle of triangle $ABC$. The median $AM$ of triangle $ABC$ intersects the incircle $\gamma$ at two points $K$ and $L$. Let the lines passing through $K$ and $L$, parallel to $BC$, intersect the incircle $\gamma$ again in two points $X$ and $Y$. Let the lines $AX$ and $AY$ intersect $BC$ again at the points $P$ and $Q$. Prove that $BP = CQ$.


$$ $$

\href{ http://www.artofproblemsolving.com/Forum/viewtopic.php?p=2#p2
}{\bf 55.} Let $ABC$ be a triangle, and $M$ an interior point such that $\angle MAB=10^\circ$, $\angle MBA=20^\circ$, $\angle MAC=40^\circ$ and $\angle MCA=30^\circ$. Prove that the triangle is isosceles.


$$ $$

\href{http://www.artofproblemsolving.com/Forum/viewtopic.php?p=448458#p448458
 }{\bf 56.} Let $ABC$ be a right-angle triangle ($AB\perp AC$). Define the middlepoint $M$ of the side $[BC]$ and the point $D\in (BC)$, $\widehat {BAD}\equiv\widehat {CAD}$. Prove that exists a point $P\in (AD)$ so that $PB\perp PM$ and $PB=PM$ if and only if $AC=2\cdot AB$ and in this case $\frac{PA}{PD}=\frac 35$.


$$ $$

\href{ http://www.artofproblemsolving.com/Forum/viewtopic.php?p=741369#p741369
}{\bf 57.} Consider a convex pentagon $ ABCDE$ such that
$$ \angle BAC = \angle CAD = \angle DAE \quad \quad  \angle ABC = \angle ACD = \angle ADE
$$
Let $ P$ be the point of intersection of the lines $ BD$ and $ CE$. Prove that the line $ AP$ passes through the midpoint of the side $ CD$.


\href{ http://www.artofproblemsolving.com/Forum/viewtopic.php?p=2111860#p2111860
}{\bf 58.} The perimeter of triangle $ABC$ is equal to $3+2\sqrt3$. In the coordinate plane, any triangle congruent to triangle $ABC$ has at least one lattice point in its interior or on its sides. Prove that triangle $ABC$ is equilateral.


$$ $$

\href{ http://www.artofproblemsolving.com/Forum/viewtopic.php?p=20670#p20670
}{\bf 59.} Let $ ABC$ be a triangle inscribed in a circle of radius $ R$, and let $ P$ be a point in the interior of triangle $ ABC$. Prove that
$$ \frac {PA}{BC^{2}} + \frac {PB}{CA^{2}} + \frac {PC}{AB^{2}}\ge \frac {1}{R}.
$$


$$ $$

\href{http://www.artofproblemsolving.com/Forum/viewtopic.php?p=794452#p794452
 }{\bf 60.} Show that the plane cannot be represented as the union of the inner regions of a finite number of parabolas.

$$ $$


\href{ http://www.artofproblemsolving.com/Forum/viewtopic.php?p=960160#p960160
}{\bf 61.} Let $ ABCD$ be a circumscriptible quadrilateral, let $ \{O\} = AC \cap BD$, and let $ K$, $ L$, $ M$, and $ N$ be the feet of the perpendiculars from the point $ O$ to the sides $ AB$, $ BC$, $ CD$, and $ DA$. Prove that: $ \frac {1}{\left|OK\right|} + \frac {1}{\left|OM\right|} = \frac {1}{\left|OL\right|} + \frac {1}{\left|ON\right|}$.


$$ $$

\href{ http://www.artofproblemsolving.com/Forum/viewtopic.php?p=515828&#p515828
}{\bf 62.} Let a triangle $ABC$ . At the extension of the sides  $BC$ (to C) ,$CA$ (to A) , $AB$ (to B) we take points  $D,E,F$  such that  $CD=AE=BF$ .
Prove that if the triangle $DEF$  is   equilateral  then $ABC$ is also equilateral.


$$ $$

\href{http://www.artofproblemsolving.com/Forum/viewtopic.php?p=744569#p744569
 }{\bf 63.} Given triangle $ABC$, incenter $I$, incircle of triangle $IBC$ touch $IB,IC$ at $I_{a},I_{a}'$ resp similar we have $I_{b},I_{b}',I_{c},I_{c}'$ the lines $I_{b}I_{b}'\cap I_{c}I_{c}'=\{A'\}$ similarly we have $B',C'$ prove that two triangle $ABC,A'B'C'$ are perspective.

$$ $$


\href{ http://www.artofproblemsolving.com/Forum/viewtopic.php?p=715139#p715139
}{\bf 64.} Let $ AA_{1},BB_{1},CC_{1}$ be the altitudes in acute triangle $ ABC$, and let $ X$ be an arbitrary point. Let $ M,N,P,Q,R,S$ be the feet of the perpendiculars from $ X$ to the lines $ AA_{1},BC,BB_{1},CA,CC_{1},AB$. Prove that $ MN,PQ,RS$ are concurrent.


$$ $$

\href{http://www.artofproblemsolving.com/Forum/viewtopic.php?p=558585#p558585
 }{\bf 65.} Let  $ABC$ be a triangle and let $X,Y$ and $Z$ be points on the sides $[BC],[CA]$ and $[AB]$, respectively, such that $AX=BY=CZ$ and $BX=CY=AZ.$ Prove that triangle $ABC$ is equilateral.


$$ $$

\href{ http://www.artofproblemsolving.com/Forum/viewtopic.php?p=124095#p124095
}{\bf 66.} Let $P$ and $P'$ be two isogonal conjugate points with respect to triangle $ABC.$ Let the lines $AP, BP, CP$ meet the lines $BC, CA, AB$ at the points $A', B', C'$, respectively.
Prove that the reflections of the lines $AP', BP', CP'$ in the lines $B'C', C'A', A'B'$ concur.


$$ $$

\href{http://www.artofproblemsolving.com/Forum/viewtopic.php?p=99759#p99759
 }{\bf 67.} In a convex quadrilateral $ABCD$, the diagonal $BD$ bisects neither the angle $ABC$ nor the angle $CDA$. The point $P$ lies inside $ABCD$ and satisfies

$\ angle PBC=\angle DBA\quad and \quad \angle PDC=\angle BDA. $

Prove that $ABCD$ is a cyclic quadrilateral if and only if $AP=CP$.


$$ $$

\href{ http://www.artofproblemsolving.com/Forum/viewtopic.php?p=564645#p564645
}{\bf 68.} Let the tangents to the circumcircle of a triangle $ABC$ at the vertices $B$ and $C$ intersect each other at a point $X.$ Then, the line $AX$ is the A-symmedian of triangle $ABC.$


$$ $$

\href{ http://www.artofproblemsolving.com/Forum/viewtopic.php?p=564645#p564645
}{\bf 69.} Let the tangents to the circumcircle of a triangle $ABC$ at the vertices $B$ and $C$ intersect each other at a point $X,$ and let $M$ be the midpoint of the side $BC$ of triangle $ABC.$ Then, $AM=AX\cdot\left|\cos A\right|$ (we don't use directed angles here).


$$ $$

\href{http://www.artofproblemsolving.com/Forum/viewtopic.php?p=529958#p529958
 }{\bf 70.} Let $ABC$ be an equilateral triangle (i. e., a triangle which satisfies $BC=CA=AB$). Let $M$ be a point on the side $BC$, let $N$ be a point on the side $CA$, and let $P$ be a point on the side $AB$, such that $S\left(ANP\right)=S\left(BPM\right)=S\left(CMN\right)$, where $S\left(XYZ\right)$ denotes the area of a triangle $XYZ$.
Prove that $\triangle ANP\cong\triangle BPM\cong\triangle CMN$.


$$ $$

\href{ http://www.artofproblemsolving.com/Forum/viewtopic.php?p=512720#p512720
}{\bf 71.}  Let $ABCD$ be a parallelogram. A variable line $g$ through the vertex $A$ intersects the rays $BC$ and $DC$ at the points $X$ and $Y$, respectively. Let $K$ and $L$ be the $A$-excenters of the triangles $ABX$ and $ADY$. Show that the angle $\measuredangle KCL$ is independent of the line $g$.










$$ $$

\href{http://www.artofproblemsolving.com/Forum/viewtopic.php?p=477064#p477064
 }{\bf 72.} Triangle $QAP$ has the right angle at $A.$ Points $B$ and $R$ are chosen on the segments $PA$ and $PQ$ respectively so that $BR$ is parallel to $AQ.$ Points $S$ and $T$ are on $AQ$ and $BR$ respectively and $AR$ is perpendicular to $BS$, and $AT$ is perpendicular to $BQ.$ The intersection of $AR$ and $BS$ is $U,$ The intersection of $AT$ and $BQ$ is $V.$ Prove that

(i) the points $P, S$ and $T$ are collinear;

(ii) the points $P, U$ and $V$ are collinear.


$$ $$

\href{http://www.artofproblemsolving.com/Forum/viewtopic.php?p=501529#p501529
 }{\bf 73.} Let $ABC$ be a triangle and $m$ a line which intersects the sides $AB$ and $AC$ at interior points $D$ and $F$, respectively, and intersects the line $BC$ at a point $E$ such that $C$ lies between $B$ and $E$. The parallel lines from the points $A$, $B$, $C$ to the line $m$ intersect the circumcircle of triangle $ABC$ at the points $A_1$, $B_1$  and $C_1$, respectively (apart from $A$, $B$, $C$). Prove that the lines $A_1E$ , $B_1F$  and  $C_1D$  pass through the same point.

$$ $$


\href{ http://www.artofproblemsolving.com/Forum/viewtopic.php?p=457871#p457871
}{\bf 74.} Let $H$ is the orthocentre of triangle $ABC$. $X$ is an arbitrary point in the plane. The circle with diameter $XH$ again meets lines $AH, BH, CH$ at a points $A_1, B_1, C_1$, and  lines $AX, BX, CX$ at a points $A_2, B_2, C_2$, respectively.  Prove that the lines $A_1A_2, B_1B_2, C_1C_2$ meet at same point.

$$ $$


\href{ http://www.artofproblemsolving.com/Forum/viewtopic.php?p=279551#p279551
}{\bf 75.} Determine the nature of a triangle $ABC$ such that the incenter lies on $HG$ where $H$ is the orthocenter and $G$
 is the centroid of the triangle $ABC$.

$$ $$


\href{http://www.artofproblemsolving.com/Forum/viewtopic.php?p=330584#p330584
 }{\bf 76.} $ABC$ is a triangle. $D$ is a point on line $AB.$ $(C)$ is the in circle of triangle $BDC$. Draw a line which is parallel to the bisector of angle $ADC,$ And goes through $I$, the incenter of $ABC$ and this line is tangent to circle $(C)$. Prove that $AD=BD.$

$$ $$


\href{ http://www.artofproblemsolving.com/Forum/viewtopic.php?p=347135#p347135
}{\bf 77.} Let $M, N$ be the midpoints of the sides $BC$ and $AC$ of $\triangle ABC$, and $BH$ be its altitude. The line through $M$, perpendicular to the bisector of $\angle HMN$, intersects the line $AC$ at point $P$ such that $HP = \frac{1}{2}(AB+BC)$ and $\angle HMN = 45$. Prove that $ABC$ is isosceles.






$$ $$

\href{http://www.artofproblemsolving.com/Forum/viewtopic.php?p=347236#p347236
 }{\bf 78.} Points $D,E,F$ are on the sides $BC, CA$ and $AB$, respectively which satisfy $EF || BC$, $D_1$ is a point on $BC,$ Make $D_1E_1 || D_E, D_1F_1 || DF$ which intersect $AC$ and $AB$ at $E_1$ and $F_1$, respectively. Make $\triangle PBC \sim \triangle DEF$ such that $P$ and $A$ are on the same side of $BC.$ Prove that $E, E_1F_1, PD_1$ are concurrent.


$$ $$

\href{ http://www.artofproblemsolving.com/Forum/viewtopic.php?p=364939#p364939}{\bf 79.} Let $ABCD$ be a rectangle. We choose 4 points like $P,M,N$ and $Q$ on $AB,BC,CD$ and $DA$. Prove that the perimeter of $PMNQ$ is at least two times the diameter of $ABCD$.

$$ $$


\href{ http://www.artofproblemsolving.com/Forum/viewtopic.php?p=363888#p363888
}{\bf 80.} In the following, the abbreviation $g \cap h$ will mean the point of intersection of two lines $g$ and $h$.

Let $ABCDE$ be a convex pentagon. Let $A^{‘}=BD\cap CE$, $B^{‘}=CE\cap DA$, $C^{‘}=DA\cap EB$, $D^{‘}=EB\cap AC$ and $E^{‘}=AC\cap BD$. Furthermore, let $A^{‘‘}=AA^{‘}\cap EB$, $B^{‘‘}=BB^{‘}\cap AC$, $C^{‘‘}=CC^{‘}\cap BD$, $D^{‘‘}=DD^{‘}\cap CE$ and $E^{‘‘}=EE^{‘}\cap DA$.

Prove that:
$$ \frac{EA^{‘‘}}{A^{‘‘}B}\cdot\frac{AB^{‘‘}}{B^{‘‘}C}\cdot\frac{BC^{‘‘}}{C^{‘‘}D}\cdot\frac{CD^{‘‘}}{D^{‘‘}E}\cdot\frac{DE^{‘‘}}{E^{‘‘}A}=1.  $$


$$ $$

\href{ http://www.artofproblemsolving.com/Forum/viewtopic.php?p=361066#p361066
}{\bf 81.}  Let $ABC$ be a triangle. The its incircle $i=C(I,r)$ touches the its sides in the points $D\in (BC),E\in (CA),F\in (AB)$ respectively. I note the second intersections $M,N,P$ of the lines $AI,BI,CI$ respectively with the its circumcircle $e=C(O,R)$. Prove that the lines $MD,NE,PF$ are concurrently.

{\bf Remark.}  If the points $A',B',C'$ are the second intersections of the lines $AO,BO,CO$ respectively with the circumcircle $e$ then the points $U\in MD\cap A'I,V\in NE\cap B'I,V\in PF\cap C'I$ belong to the circumcircle $w$.


$$ $$

\href{ http://www.artofproblemsolving.com/Forum/viewtopic.php?p=331636#p331636
}{\bf 82.} let $ABC$ be an acute triangle with $\angle BAC>\angle BCA$, and let $D$ be a point on side $AC$ such that
$|AB|=|BD|$. Furthermore, let $F$ be a point on the circumcircle of triangle $ABC$
such that line $FD$ is perpendicular to side $BC$ and points $F,B$ lie on different sides of line $AC$.
Prove that line $FB$ is perpendicular to side $AC.$


$$ $$

\href{ http://www.artofproblemsolving.com/Forum/viewtopic.php?p=245463#p245463
}{\bf 83.}  Let $ABC$ be a triangle with orthocenter $H$, incenter $I$ and centroid $S$, and let $d$ be the diameter of the circumcircle of triangle $ABC$.

Prove the inequality

$9\cdot HS^2+4\left(AH\cdot AI+BH\cdot BI+CH\cdot CI\right)\geq 3d^2$,

and determine when equality holds.


$$ $$

\href{ http://www.artofproblemsolving.com/Forum/viewtopic.php?p=336205#p336205
}{\bf 84.}  Let $ABC$ be a triangle. A circle passing through $A$ and $B$ intersects segments $AC$ and $BC$ at $D$ and $E$, respectively. Lines $AB$ and $DE$ intersect at $F$, while lines $BD$ and $CF$ intersect at $M$. Prove that $MF = MC$ if and only if $MB\cdot MD = MC^2$.


$$ $$

\href{http://www.artofproblemsolving.com/Forum/viewtopic.php?p=329277#p329277
 }{\bf 85.} $ABC$ inscribed triangle in circle $(O,R)$. At $AB$ we take point $C'$ such that $AC=AC'$ and at $AC$ we take point $B'$ such that $AB'=AB$. The segment $B'C'$ intersects the circle at $E,D$ respectively and and it intersects $BC$ at $M$.
Prove that when the point $A$ moves on the arc $BAC$ the $AM$ pass from a standard point.

$$ $$


\href{http://www.artofproblemsolving.com/Forum/viewtopic.php?p=320049#p320049
 }{\bf 86.} In an acute-angled triangle $ABC$, we consider the feet $H_a$ and $H_b$ of the altitudes from $A$ and $B$, and the intersections $W_a$ and $W_b$ of the angle bisectors from $A$ and $B$ with the opposite sides $BC$ and $CA$ respectively. Show that the centre of the incircle $I$ of triangle $ABC$ lies on the segment $H_aH_b$ if and only if the centre of the circumcircle $O$ of triangle $ABC$ lies on the segment $W_aW_b$.


$$ $$

\href{ http://www.artofproblemsolving.com/Forum/viewtopic.php?p=316789#p316789
}{\bf 87.} Let $ABC$ be a triangle and $O$ a point in its plane. Let the lines $BO$ and $CO$ intersect the lines $CA$ and $AB$ at the points $M$ and $N,$ respectively. Let the parallels to the lines $CN$ and $BM$ through the points $M$ and $ N$ intersect each other at $E,$ and let the parallels to the lines $CN$ and $BM$ through the points $B$ and $C$ intersect each other at $F.$


$$ $$

\href{ http://www.artofproblemsolving.com/Forum/viewtopic.php?p=310011#p310011
}{\bf 88.}  In space, given a right-angled triangle $ABC$ with the right angle at $A,$ and given a point $D$ such that the line $CD$ is perpendicular to the plane $ABC.$ Denote $d = AB, h = CD$, $\alpha=\measuredangle DAC$ and $\beta=\measuredangle DBC$. Prove that $h=\frac{d\tan\alpha\tan\beta}{\sqrt{\tan^2\alpha-\tan^2\beta}}$.


$$ $$

\href{http://www.artofproblemsolving.com/Forum/viewtopic.php?p=311023#p311023
 }{\bf 89.} A triangle $ABC$ is given in a plane. The internal angle bisectors of the angles $A, B, C$ of this triangle ABC intersect the sides $BC, CA, AB$ at $A', B', C'.$ Let $P$ be the point of intersection of the angle bisector of the angle $A$ with the line $B'C'.$ The parallel to the side $BC$ through the point $P$ intersects the sides $AB $and $AC$ in the points $M$ and $N.$ Prove that $2\cdot MN = BM + CN$.

$$ $$


\href{http://www.artofproblemsolving.com/Forum/viewtopic.php?p=307795#p307795
 }{\bf 90.} A triangle $ABC$ has the sidelengths $a$, $b$, $c$ and the angles $A$, $B$, $C$, where $a$ lies opposite to $A$, where $b$ lies opposite to $B$, and $c$ lies opposite to $C$.
If $a\left(1-2 \cos A\right)+b\left(1-2 \cos B\right)+c\left(1-2 \cos C\right) = 0$, then prove that the triangle $ABC$ is equilateral.

$$ $$
¬

\href{ http://www.artofproblemsolving.com/Forum/viewtopic.php?p=376521#p376521
}{\bf 91.}  Circles $C(O_1)$ and $C(O_2)$ intersect at points $A$, $B$. $CD$ passing through point $O_1$ intersects $C(O_1)$ at point $D$ and tangents $C(O_2)$ at point $C$. $AC$ tangents $C(O_1)$ at $A$. Draw $AE \bot CD$, and $AE$ intersects $C(O_1)$ at $E$. Draw $AF \bot DE$, and $AF$ intersects $DE$ at $F$. Prove that $BD$ bisects $AF$.


\href{http://www.artofproblemsolving.com/Forum/viewtopic.php?p=292643#p292643
 }{\bf 92.} In a triangle $ABC$, let $A_{1}$, $B_{1}$, $C_{1}$ be the points where the excircles touch the sides $BC$, $CA$ and $AB$ respectively. Prove that $A A_{1}$, $B B_{1}$ and $C C_{1}$ are the sidelenghts of a triangle.

$$ $$


\href{http://www.artofproblemsolving.com/Forum/viewtopic.php?p=213011#p213011
 }{\bf 93.} Let $ABC$ be an acute-angled triangle, and let $P$ and $Q$ be two points on its side $BC$. Construct a point $C_{1}$ in such a way that the convex quadrilateral $APBC_{1}$ is cyclic, $QC_{1}\parallel CA$, and the points $C_{1}$ and $Q$ lie on opposite sides of the line $AB$. Construct a point $B_{1}$ in such a way that the convex quadrilateral $APCB_{1}$ is cyclic, $QB_{1}\parallel BA$, and the points $B_{1}$ and $Q$ lie on opposite sides of the line $AC$.  

Prove that the points $B_{1}$, $C_{1}$, $P$, and $Q$ lie on a circle.

$$ $$


\href{ http://www.artofproblemsolving.com/Forum/viewtopic.php?p=191489#p191489
}{\bf 94.} Let $ABCD$ be an arbitrary quadrilateral. The bisectors of external angles $A$ and $C$ of the quadrilateral intersect at $P$; the bisectors of external angles $B$ and $D$ intersect at $Q$. The lines $AB$ and $CD$ intersect at $E$, and the lines $BC$ and $DA$ intersect at $F$. Now we have two new angles: $E$ (this is the angle $\angle{AED}$) and $F$ (this is the angle $\angle{BFA}$). We also consider a point $R$ of intersection of the external bisectors of these angles. Prove that the points $P$, $Q$ and $R$ are collinear.

$$ $$


\href{http://www.artofproblemsolving.com/Forum/viewtopic.php?p=190783#p190783
 }{\bf 95.} Let I be the incenter in triangle $ABC$ and let triangle $A_1B_1C_1$ be its medial triangle (i.e. $A_1$ is the midpoint of $BC$, etc.).  Prove that the centers of Euler's nine-point circles of triangle $BIC, CIA, AIB$ lie on the angle bisectors of the medial triangle $A_1B_1C_1$.



\href{http://www.artofproblemsolving.com/Forum/viewtopic.php?p=190783#p190783
 }{\bf 96.} Consider three circles equal radii R that have a common point $H.$  They intersect also two by two in three other points different than $H,$ denoted $A, B, C.$  Prove that the circumradius of triangle $ABC$ is also $R.$

$$ $$


\href{ http://www.artofproblemsolving.com/Forum/viewtopic.php?p=6461#p6461
}{\bf 97.} Three congruent circles $G_{1}$, $G_{2}$, $G_{3}$ have a common point $P$.
Further, define $G_{2}\cap G_{3}=\left\{A,\ P\right\}$, $G_{3}\cap G_{1}=\left\{B,\ P\right\}$, $G_{1}\cap G_{2}=\left\{C,\ P\right\}$.
{\bf 1)}  Prove that the point $P$ is the orthocenter of triangle $ABC$.
{\bf 2)}  Prove that the circumcircle of triangle $ABC$ is congruent to the given circles $G_{1}$, $G_{2}$, $G_{3}$.


$$ $$

\href{http://www.artofproblemsolving.com/Forum/viewtopic.php?p=235599#p235599
 }{\bf 98.} Let $ABXY$ be a convex trapezoid such that $BX \parallel AY.$ We call C the midpoint of its side $XY,$ and we denote by $P$ and $Q$ the midpoints of the segments $BC$ and $CA,$ respectively. Let the lines $XP$ and $YQ$ intersect at a point $N.$ Prove that the point $N$ lies in the interior or on the boundary of triangle $ ABC $ if and only if $\frac13\leq\frac{BX}{AY}\leq 3$.


$$ $$

\href{ http://www.artofproblemsolving.com/Forum/viewtopic.php?p=117045#p117045
}{\bf 99.} Let $P$ be a fixed point on a conic, and let $M,N$ be variable points on that same conic s.t. $PM\perp PN$. Show that $MN$ passes through a fixed point.

$$ $$

\href{ http://www.artofproblemsolving.com/Forum/viewtopic.php?p=239221#p239221
}{\bf 100.} A triangle $ABC$ is given. Let $L$ be its Lemoine point and $F$ its Fermat (Torricelli) point. Also, let $H$ be its orthocenter and $O$ its circumcenter. Let $l$ be its Euler line and $l'$ be a reflection of $l$ with respect to the line $AB$. Call $D$ the intersection of $l'$ with the circumcircle different from $H'$ (where $H'$ is the reflection of $H$ with respect to the line $AB$), and $E$ the intersection of the line $FL$ with $OD$. Now, let $G$ be a point different from $H$ such that the pedal triangle of $G$ is similar to the cevian triangle of $G$ (with respect to triangle $ABC$). Prove that angles $ACB$ and $GCE$ have either common or perpendicular bisectors.

$$ $$

\href{http://www.artofproblemsolving.com/Forum/viewtopic.php?p=230527#p230527
 }{\bf 101.} Let $ABC$ be a triangle with area $S$, and let $P$ be a point in the plane. Prove that $AP+BP+CP\geq 2\sqrt[4]{3}\sqrt{S}$.


$$ $$

\href{http://www.artofproblemsolving.com/Forum/viewtopic.php?p=124968#p124968
 }{\bf 102.} Suppose $M$ is a point on the side $AB$ of triangle $ABC$ such that the incircles of triangle $AMC$ and triangle $BMC$ have the same radius. The two circles, centered at $O_1$ and $O_2$, meet $AB$ at $P$ and $Q$ respectively. It is known that the area of triangle $ABC$ is six times the area of the quadrilateral $PQO_2O_1$, determine the possible value(s) of $\frac{AC+BC}{AB}$. Justify your claim.


$$ $$

\href{ http://www.artofproblemsolving.com/Forum/viewtopic.php?p=146876#p146876
}{\bf 103.} Let $AB_{1}C_{1}$, $AB_{2}C_{2}$, $AB_{3}C_{3}$ be directly congruent equilateral triangles. Prove that the pairwise intersections of the circumcircles of triangles $AB_{1}C_{2}$, $AB_{2}C_{3}$, $AB_{3}C_{1}$ form an equilateral triangle congruent to the first three.


$$ $$

\href{ http://www.artofproblemsolving.com/Forum/viewtopic.php?p=201120#p201120
}{\bf 104.} Tried posting this in Pre-Olympiad but thought I'd get more feed back here:
For acute triangle ABC, cevians AD, BE, and CF are concurrent at P.
Prove
$\displaystyle 2\left(\frac{1}{AP}+\frac{1}{BP}+\frac{1}{CP}\right)\leq \frac{1}{PD}+\frac{1}{PE}+\frac{1}{PF}$
and determine when equality holds


$$ $$


\href{http://www.artofproblemsolving.com/Forum/viewtopic.php?p=268390#p268390
 }{\bf 105.} Given a triangle $ABC$. Let $O$ be the circumcenter of this triangle $ABC$. Let $H$, $K$, $L$ be the feet of the altitudes of triangle $ABC$ from the vertices $A$, $B$, $C$, respectively. Denote by $A_{0}$, $B_{0}$, $C_{0}$ the midpoints of these altitudes $AH$, $BK$, $CL$, respectively. The incircle of triangle $ABC$ has center $I$ and touches the sides $BC$, $CA$, $AB$ at the points $D$, $E$, $F$, respectively. Prove that the four lines $A_{0}D$, $B_{0}E$, $C_{0}F$ and $OI$ are concurrent. (When the point $O$ concides with $I$, we consider the line $OI$ as an arbitrary line passing through $O$.)


$$ $$


\href{http://www.artofproblemsolving.com/Forum/viewtopic.php?p=268044#p268044
 }{\bf 106.} Given an equilateral triangle $ABC$ and a point $M$ in the plane ($ABC$). Let $A', B', C'$ be respectively the symmetric through $M$ of $A, B, C$.

{\bf. I.} Prove that there exists a unique point $P$ equidistant from $A$ and $B'$, from $B$ and $C'$ and from $C$ and $A'$.

{\bf. II.}  Let $D$ be the midpoint of the side $AB$. When $M$ varies ($M$ does not coincide with $D$), prove that the circumcircle of triangle $MNP$ ($N$ is the intersection of the line $DM$ and $AP$) pass through a fixed point.


$$ $$


\href{ http://www.artofproblemsolving.com/Forum/viewtopic.php?p=220569#p220569
}{\bf 107.} Let $ABCD$ be a square, and $C$ the circle whose diameter is $AB.$ Let $Q$ be an arbitrary point on the segment $CD.$ We know that $QA$ meets $C$ on $E$ and $QB$ meets it on $F.$ Also $CF$ and $DE$ intersect in $M.$ show that $M$ belongs to $C.$


$$ $$


\href{ http://www.artofproblemsolving.com/Forum/viewtopic.php?p=278616#p278616
}{\bf 108.} In a triangle, let $a,b,c$ denote the side lengths and $h_a, h_b, h_c$ the altitudes to the corresponding side. Prove that $(\frac{a}{h_a})^2+(\frac{b}{h_b})^2+(\frac{c}{h_c})^2 \geq 4$


$$ $$


\href{http://www.artofproblemsolving.com/Forum/viewtopic.php?p=135082#p135082
 }{\bf 109.} Given a triangle $ABC$. A point $X$ is chosen on a side $AC$. Some circle passes through $X$, touches the side $AC$ and intersects the circumcircle of triangle $ABC$ in points $M$ and $N$ such that the segment $MN$ bisects $BX$ and intersects sides $AB$ and $BC$ in points $P$ and $Q$. Prove that the circumcircle of triangle $PBQ$ passes through a fixed point different from $B$.


$$ $$


\href{ http://www.artofproblemsolving.com/Forum/viewtopic.php?p=159507#p159507
}{\bf 110.} Let $ABC$ be an isosceles triangle with $\angle ACB=\frac\pi 2$, and let $P$ be a point inside it.

{\bf. A)} Show that $\angle PAB+\angle PBC\ge\min(\angle PCA,\angle PCB)$;

{\bf. B)} When does equality take place in the inequality above?


$$ $$


\href{ http://www.artofproblemsolving.com/Forum/viewtopic.php?p=117329#p117329
}{\bf 111.} Given a regular  tetrahedron   $ABCD$   with edge length $1$ and a point $P$ inside it.
What is the maximum value of   $\left|PA\right|+\left|PB\right|+\left|PC\right|+\left|PD\right|$.


$$ $$


\href{ http://www.artofproblemsolving.com/Forum/viewtopic.php?p=134422#p134422
}{\bf 112.} Given the tetrahedron $ABCD$ whose faces are all congruent. The vertices $A$, $B$, $C$ lie in the positive part of $x$-axis, $y$-axis, and $z$-axis, respectively, and $AB=2l-1$, $BC=2l$, $CA=2l+1$, where $l>2$. Let the volume of tetrahedron ABCD be $V\left(l\right)$.

Evaluate $$\lim_{l\to 2} \frac{V\left(l\right)}{\sqrt{l-2}}$$.


$$ $$


\href{http://www.artofproblemsolving.com/Forum/viewtopic.php?p=134462#p134462
 }{\bf 113.} Let a triangle ABC . M , N , P are the midpoints of BC, CA, AB .
a) $d_1, d_2, d_3$ are lines throughing M, N, P and dividing the perimeter of triangle ABC into halves . Prove that : $d_1, d_2, d_3$ are concurrent at K .
b) Prove that : among the ratios : $\frac{KA}{BC}, \frac{KB}{AC}, \frac{KC}{AB}$, there exists at least one ratio $\geq \frac{1}{\sqrt3}$.


$$ $$


\href{http://www.artofproblemsolving.com/Forum/viewtopic.php?p=133409#p133409
 }{\bf 114.} Given rectangle $ABCD \ (AB=a,BC=b)$  find locus of points $M$ , so that reflections of $M$  in the sides are concyclic.


$$ $$


\href{http://www.artofproblemsolving.com/Forum/viewtopic.php?p=136311#p136311
 }{\bf 115.} An incircle of a triangle $ABC$ touches it's sides $AB$, $BC$ and $CA$ at $C'$, $A'$ and $B'$ respectively. Let $M$, $N$, $K$, $L$ be midpoints of $C'A$, $B'A$, $A'C$, $B'C$ respectively. The line $A'C'$ intersects lines $MN$ and $KL$ at $E$ and $F$ respectively; lines $A'B'$ and $MN$ intersect at $P$; lines $B'C'$ and $KL$ intersect at $Q$. Let $\Omega_A$ and $\Omega_C$ be outcircles of triangles $EAP$ and $FCQ$ respectively.
a) Let $l_1$ and $l_2$ be common tangents of circles $\Omega_A$ and $\Omega_C$. Prove that the lines $l_1$, $l_2$, $EF$ and $PQ$ have a common point.
b) Let circles $\Omega_A$ and $\Omega_C$ intersect at $X$ and $Y$. Prove that the points $X$, $Y$ and $B$ lie on the line.


$$ $$





\href{http://www.artofproblemsolving.com/Forum/viewtopic.php?p=138570#p138570
 }{\bf 116.} Let two circles $\left(O_1\right)$ and $\left(O_2\right)$ cut each other at two points $A$ and $B$. Let a point $M$ move on the circle $\left(O_1\right)$. Denote by $K$ the point of intersection of the two tangents to the circle $\left(O_1\right)$ at the points $A$ and $B$. Let the line $MK$ cut the circle $\left(O_1\right)$ again at $C$. Let the line $AC$ cut the circle $\left(O_2\right)$ again at $Q$. Let the line $MA$ cut the circle $\left(O_2\right)$ again at $P$.
{\bf (a)} Prove that the line $KM$ bisects the segment $PQ$.
{\bf  (b)} When the point $M$ moves on the circle $\left(O_1\right)$, prove that the line $PQ$ passes through a fixed point.


$$ $$


\href{ http://www.artofproblemsolving.com/Forum/viewtopic.php?p=112682#p112682
}{\bf 117.} Given n balls $B_1$, $B_2$, ..., $B_n$ of radii $R_1$, $R_2$, ..., $R_n$ in space. Assume that there doesn't exist any plane separating these n balls. Then prove that there exists a ball of radius $R_1+R_2+...+R_n$ which covers all of our n balls $B_1$, $B_2$, ..., $B_n$.


$$ $$


\href{ http://www.artofproblemsolving.com/Forum/viewtopic.php?p=143442#p143442
}{\bf 118.} Let $ABC$ be a triangle, and erect three rectangles $ABB_1A_2$, $BCC_1B_2$, $CAA_1C_2$ externally on its sides $AB$, $BC$, $CA$, respectively. Prove that the perpendicular bisectors of the segments $A_1A_2$, $B_1B_2$, $C_1C_2$ are concurrent.


$$ $$


\href{ http://www.artofproblemsolving.com/Forum/viewtopic.php?p=112052#p112052
}{\bf 119.} On a line points $A,B,C,D$ are given in this order s.t. $AB=CD$. Can we find the midpoint of $BC$ using only a straightedge?


$$ $$


\href{ http://www.artofproblemsolving.com/Forum/viewtopic.php?p=143448#p143448
}{\bf 120.} Let $ABC$ be a triangle, and $D$, $E$, $F$ the points where its incircle touches the sides $BC$, $CA$, $AB$, respectively. The parallel to $AB$ through $E$ meets $DF$ at $Q$, and the parallel to $AB$ through $D$ meets $EF$ at $T$. Prove that the lines $CF$, $DE$, $QT$ are concurrent.


$$ $$


\href{ http://www.artofproblemsolving.com/Forum/viewtopic.php?p=144399#p144399
}{\bf 121.} Given the triangle $ABC.$ $I$ and $N$ are the incenter and the Nagel point of $ABC$, and $r$ is the in radius of $ABC.$ Prove that
$$IN=r \iff  a+b=3c \text{ or } b+c=3a \text{ or } c+a=3b$$


$$ $$


\href{http://www.artofproblemsolving.com/Forum/viewtopic.php?p=149725#p149725
 }{\bf 122.} The centers of three circles isotomic with the Apollonian circles of triangle $ABC$ located on a line perpendicular to the Euler line of $ABC.$


$$ $$


\href{http://www.artofproblemsolving.com/Forum/viewtopic.php?p=10533#p10533
 }{\bf 123.} Let $ABC$ be a triangle, and $M$ and $M'$ two points in its plane. Let $X$ and $X'$ be two points on the line $BC,$ let $Y$ and $Y'$ be two points on the line $CA,$ and let $Z$ and $Z'$ be two points on the line $AB.$ Assume that
$$M'X  \parallel  AM; M'Y \parallel  BM; M'Z \parallel  CM; MX' \parallel  AM'; MY' \parallel  BM'; MZ' \parallel  CM'. $$
Prove that the lines $AX, BY, CZ$ concur if and only if the lines $AX', BY', CZ'$ concur.


$$ $$


\href{http://www.artofproblemsolving.com/Forum/viewtopic.php?p=154202#p154202
 }{\bf 124.} Let's call a sextuple of points $(A, B, C, D, E, F)$ in the plane a Pascalian sextuple if and only if the points of intersection $AB\cap DE$, $BC\cap EF$ and $CD\cap FA$ are collinear.
Prove that if a sextuple of points is Pascalian, then each permutation of this sextuple is Pascalian.


$$ $$


\href{http://www.artofproblemsolving.com/Forum/viewtopic.php?p=157209#p157209
 }{\bf 125.} If $ P$ be any point on the circumcircle of a triangle $ ABC$ whose Lemoine point is $ K$, show that the line $ PK$ will cut the sides $ BC$, $ CA$, $ AB$ of the triangle in points $ X$, $ Y$, $ Z$ so that
$$ \frac{3}{{PK}}=\frac{1}{{PX}}+\frac{1}{{PY}}+\frac{1}{{PZ}}$$
where the segments are directed.


$$ $$


\href{http://www.artofproblemsolving.com/Forum/viewtopic.php?p=121765#p121765
 }{\bf 126.} Given four distinct points $A_1,A_2,B_1,B_2$ in the plane, show that if every circle through $A_1,A_2$ meets every circle through $B_1,B_2$, then $A_1,A_2,B_1,B_2$ are either collinear or concyclic.


$$ $$


\href{ http://www.artofproblemsolving.com/Forum/viewtopic.php?p=131325#p131325
}{\bf 127.} $ABCD$ is a convex quadrilateral s.t. $AB$ and $CD$ are not parallel. The circle through $A,B$ touches $CD$ at $X$, and a circle through $C,D$ touches $AB$ at $Y$. These two circles intersect in $U,V$. Show that $AD\|BC\iff$ $UV$ bisects $XY$.


$$ $$


\href{ http://www.artofproblemsolving.com/Forum/viewtopic.php?p=142514#p142514
}{\bf 128.} Given $R,r$, construct circles with radi $R,r$ s.t. the distance between their centers is equal to their common chord.


$$ $$


\href{ http://www.artofproblemsolving.com/Forum/viewtopic.php?p=142902#p142902
}{\bf 129.} Construct triangle $ABC$, given the midpoint $M$ of $BC$, the midpoint $N$ of $AH$ ($H$ is the orthocenter), and the point $A'$ where the incircle touches $BC$.


$$ $$


\href{ http://www.artofproblemsolving.com/Forum/viewtopic.php?p=157721#p157721
}{\bf 130.} Let $A',B',C'$ be the reflections of the vertices $A,B,C$ in the sides $BC,CA,AB$ respectively. Let $O$ be the circumcenter of $ABC$. Show that the circles $(AOA'),(BOB'),(COC')$ concur again in a point $P$, which is the inverse in the circumcircle of the isogonal conjugate of the nine-point center.


$$ $$


\href{ http://www.artofproblemsolving.com/Forum/viewtopic.php?p=159507#p159507
}{\bf 131.} Let $ABC$ be an isosceles triangle with $\angle ACB=\frac\pi 2$, and let $P$ be a point inside it.

a) Show that $\angle PAB+\angle PBC\ge\min(\angle PCA,\angle PCB)$;

b) When does equality take place in the inequality above?


$$ $$


\href{ http://www.artofproblemsolving.com/Forum/viewtopic.php?p=189522#p189522
}{\bf 132.} Let $S$ be the set of all polygonal surfaces in the plane (a polygonal surface is the interior together with the boundary of a non-self-intersecting polygon; the polygons do not have to be convex). Show that we can find a function $f:S\to (0,1)$ s.t. if $S_1,S_2,S_1\cup S_2\in S$ and the interiors of $S_1,S_2$ are disjoint, then $f(S_1\cup S_2)=f(S_1)+f(S_2)$.


$$ $$


\href{http://www.artofproblemsolving.com/Forum/viewtopic.php?p=204466#p204466
 }{\bf 133.} Let $A'B'C'$ be the orthic triangle of $ABC$, and let $A'',B'',C''$ be the orthocenters of $AB'C',A'BC',A'B'C$ respectively. Show that $A'B'C',A''B''C''$ are homothetic.


$$ $$

\href{ http://www.artofproblemsolving.com/Forum/viewtopic.php?p=215657#p215657
}{\bf 134.} Let $O$ be the midpoint of a chord $AB$ of an ellipse. Through $O$, we draw another chord $PQ$ of the ellipse. The tangents in $P,Q$ to the ellipse cut $AB$ in $S,T$ respectively. Show that $AS=BT$.


$$ $$


\href{http://www.artofproblemsolving.com/Forum/viewtopic.php?p=220177#p220177
 }{\bf 135.} Given a parallelogram $ABCD$ with $AB<BC$, show that the circumcircles of the triangles $APQ$ share a second common point (apart from $A$) as $P,Q$ move on the sides $BC,CD$ respectively s.t. $CP=CQ$.


$$ $$


\href{ http://www.artofproblemsolving.com/Forum/viewtopic.php?p=220179#p220179
}{\bf 136.} We have an acute-angled triangle $ABC$, and $AA',BB'$ are its altitudes. A point $D$ is chosen on the arc $ACB$ of the circumcircle of $ABC$. If $P=AA'\cap BD,Q=BB'\cap AD$, show that the midpoint of $PQ$ lies on $A'B'$.


$$ $$


\href{http://www.artofproblemsolving.com/Forum/viewtopic.php?p=229579#p229579
 }{\bf 137.} Let $(I),(O)$ be the incircle, and, respectiely, circumcircle of $ABC$. $(I)$ touches $BC,CA,AB$ in $D,E,F$ respectively. We are also given three circles $\omega_a,\omega_b,\omega_c$, tangent to $(I),(O)$ in $D,K$ (for $\omega_a$), $E,M$ (for $\omega_b$), and $F,N$ (for $\omega_c$).

a) Show that $DK,EM,FN$ are concurrent in a point $P$;

b) Show that the orthocenter of $DEF$ lies on $OP$.


$$ $$


\href{http://www.artofproblemsolving.com/Forum/viewtopic.php?p=260264#p260264
 }{\bf 138.} Given four points $A,B,C,D$ in the plane and another point $P$, the polars of $P$ wrt the conics passing through $A,B,C,D$ pass through a fixed point (well, unless $P$ is one of $AB\cap CD,AD\cap BC,AC\cap BD$, in which case the polar is fixed).


$$ $$


\href{ http://www.artofproblemsolving.com/Forum/viewtopic.php?p=319718#p319718
}{\bf 139.} Prove that if the hexagon $A_1A_2A_3A_4A_5A_6$ has all sides of length $\le 1$, then at least one of the diagonals $A_1A_4,A_2A_5,A_3A_6$ has length $\le 2$.


$$ $$


\href{ http://www.artofproblemsolving.com/Forum/viewtopic.php?p=320556#p320556
}{\bf 140.} Find the largest $k>0$ with the property that for any convex polygon of area $S$ and any line $\ell$ in the plane, we can inscribe a triangle with area $\ge kS$ and a side parallel to $\ell$ in the polygon.


$$ $$


\href{ http://www.artofproblemsolving.com/Forum/viewtopic.php?p=324872#p324872
}{\bf 141.}  Given a finite number of parallel segments in the plane s.t. for each three there is a line intersecting them, prove that there is a line intersecting all the segments.


$$ $$


\href{ http://www.artofproblemsolving.com/Forum/viewtopic.php?p=397953#p397953
}{\bf 142.} Let $A_0A_1\ldots A_n$ be an $n$-dimensional simplex, and let $r,R$ be its inradius and circumradius, respectively. Prove that $R\ge nr$.


$$ $$


\href{ http://www.artofproblemsolving.com/Forum/viewtopic.php?p=415477#p415477
}{\bf 143.} Find those $n\ge 2$ for which the following holds:

For any $n+2$ points $P_1,\ldots,P_{n+2}\in\mathbb R^n$, no three on a line, we can find $i\ne j\in\overline{1,n+2}$ such that $P_iP_j$ is not an edge of the convex hull of the points $P_i$.


$$ $$

\href{ http://www.artofproblemsolving.com/Forum/viewtopic.php?p=434449#p434449
}{\bf 144.} Given $n+1$ convex polytopes in $\mathbb R^n$, prove that the following two assertions are equivalent:

(a) There is no hyperplane which meets all $n+1$ polytopes;

(b) Every polytope can be separated from the other $n$ by a hyperplane.


$$ $$


\href{ http://www.artofproblemsolving.com/Forum/viewtopic.php?p=475049#p475049
}{\bf 145.} Find those convex polygons which can be covered by $3$ strictly smaller homothetic images of themselves (i.e. images through homothecies with ratio in the interval $(0,1)$).


$$ $$


\href{ http://www.artofproblemsolving.com/Forum/viewtopic.php?p=20670#p20670 }{\bf 146.} Let $ ABC$ be a triangle inscribed in a circle of radius $ R$, and let $ P$ be a point in the interior of triangle $ ABC$. Prove that
$$\frac {PA}{BC^{2}} \plus{} \frac {PB}{CA^{2}} \plus{} \frac {PC}{AB^{2}}\ge \frac {1}{R}.$$

$$ $$


\href{ http://www.artofproblemsolving.com/Forum/viewtopic.php?p=773#p773}{\bf 147.}  There is an odd number of soldiers, the distances between all of them being all distinct, which are  training as follows: each one of them is looking at the one closest to them. Show that there is a soldier which nobody is looking at.

$$ $$


\href{http://www.artofproblemsolving.com/Forum/viewtopic.php?p=181685#p181685 }{\bf 148.} Let $H$ be the orthocenter of the acute triangle $ABC$. Let $BB'$ and $CC'$ be altitudes of the triangle ($B^{\prime} \in AC$, $C^{\prime} \in AB$). A variable line $\ell$ passing through $H$ intersects the segments $[BC']$ and $[CB']$ in $M$ and $N$. The perpendicular lines of $\ell$ from $M$ and $N$ intersect $BB'$ and $CC'$ in $P$ and $Q$. Determine the locus of the midpoint of the segment $[ PQ]$.

$$ $$


\href{http://www.artofproblemsolving.com/Forum/viewtopic.php?p=117715#p117715 }{\bf 149.} Show that there are no regular polygons with more than $4$ sides inscribed in an ellipse.

$$ $$


\href{http://www.artofproblemsolving.com/Forum/viewtopic.php?p=117345#p117345 }{\bf 150.} Given a cyclic $2n$-gon with a fixed circumcircle such that $2n-1$ of its sides pass through $2n-1$ fixed point lying on a line $\ell$, show that the $2n$’th side also passes through a fixed point on $\ell$.


$$ \huge \text{ END.}$$

\end{document}
