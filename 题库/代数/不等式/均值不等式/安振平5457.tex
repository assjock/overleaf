\documentclass[]{article}


%导入宏包
\usepackage{ctex}
\usepackage{graphics}
\usepackage{amsmath,amssymb}
\usepackage{amsthm}
\usepackage{pst-blur}
\usepackage{pstricks-add}
\usepackage{pstricks}
\usepackage{pstricks-pdf}
\usepackage{pgf,tikz,pgfplots}
\usepackage{mathrsfs}
\usepackage{xcolor}
\usepackage{eso-pic}
%\usepackage{background}
\usepackage{xeCJK}
    \xeCJKsetup{CJKmath=true}
%\usepackage{wallpaper}


%opening
\title{}
\author{}
\date{}
\setlength\parindent{0em}

\begin{document}
\maketitle
\section{安振平5457}{
已知$a,b,c \in \mathbf{R},a+b+c=3,a^2+b^2+c^2=5$,求证:$|(a-b)(b-c)(c-a)|\leq 2$.

\begin{proof}
不妨令$a\geq b\geq c$.\\
\begin{equation}\nonumber
    \Longleftrightarrow (a-b)(b-c)(a-c)\leq 2
\end{equation}
由均值不等式知:
\begin{equation}\nonumber
    (a-b)(b-c)(a-c)\leq \frac{(a-c)^3}{4}
\end{equation}
只需证
\begin{equation}\label{daizheng}
    \frac{(a-c)^3}{4}\leq 2
\end{equation}
转换条件,得到
\begin{equation}\label{tiaojian}
    a^2+c^2+(3-a-c)^2=5
\end{equation}
用$a_1=a+1,c_1=c+1$替换\eqref{tiaojian}的字母,得到
\begin{equation}\label{tiaojianbianxing}
    (a_0+1)^2+(c_0+1)^2+(1-x-y)^2=5
\end{equation}
再用$\displaystyle a_2=\frac{\sqrt{2}}{2}(a_1-c_1),c_2=\frac{\sqrt{2}}{2}(a_1+c_1)$替换\eqref{tiaojianbianxing}中字母,整理后得到
\begin{equation}
    \frac{a^2}{2}+\frac{3y^2}{2}=1
\end{equation}
这是一个椭圆.\\
逆向过程知,条件所描述的椭圆长轴顶点$(2,0)(0,2)$,长轴为$a+c=2$,则易知$a-c\leq 2$
代入即得
\begin{equation}\nonumber
    \frac{(a-c)^3}{4}\leq 2
\end{equation}
\end{proof}
}
\end{document}