\documentclass[]{article}


%导入宏包
\usepackage{ctex}
\usepackage{graphics}
\usepackage{amsmath,amssymb}
\usepackage{amsthm}
\usepackage{pst-blur}
\usepackage{pstricks-add}
\usepackage{pstricks}
\usepackage{pstricks-pdf}
\usepackage{pgf,tikz,pgfplots}
\usepackage{mathrsfs}
\usepackage{xcolor}
\usepackage{eso-pic}
%\usepackage{background}
\usepackage{xeCJK}
    \xeCJKsetup{CJKmath=true}
%\usepackage{wallpaper}

%opening
\title{}
\author{}
\date{}
\setlength\parindent{0em}

\begin{document}
\maketitle
\section{2018CMO T5}{
给定的正整数$n$,在$n\times n$的方格表中的每一格中填一个整数.对此方格表进行如下的操作:选取一个小格,将此格所在的行和列中的$2n+1$个格的每个数格加$1$.求最大的整数$N=N(n)$,使得无论初始时方格表中的数是多少,总可以经过一系列上述操作,使得方格表中至少有$N$个偶数.
%\begin{proof}{}
%\end{proof}
}
\end{document}