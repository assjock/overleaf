\documentclass[]{article}


%导入宏包
\usepackage{ctex}
\usepackage{graphics}
\usepackage{amsmath,amssymb}
\usepackage{amsthm}
\usepackage{pst-blur}
\usepackage{pstricks-add}
\usepackage{pstricks}
\usepackage{pstricks-pdf}
\usepackage{pgf,tikz,pgfplots}
\usepackage{mathrsfs}
\usepackage{xcolor}
\usepackage{eso-pic}
%\usepackage{background}
\usepackage{xeCJK}
    \xeCJKsetup{CJKmath=true}
%\usepackage{wallpaper}

%opening
\title{}
\author{}
\date{}
\setlength\parindent{0em}

\begin{document}
\maketitle
\section{2018CMO T6}{
将2018个点$P_1,P_2,…,P_{2018}$(点的位置可相同)放置在一个给定的正五边形的内部或边界上.求所有放置方式,使得
\begin{equation}\nonumber
    S=\sum\limits_{1\leq i<j\leq 2018}\left| P_i P_j \right|^2
\end{equation}
达到最大,并证明结论.
%\begin{proof}{}
%\end{proof}
}
\end{document}