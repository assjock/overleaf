\documentclass[]{article}


%导入宏包
\usepackage{ctex}
\usepackage{graphics}
\usepackage{amsmath,amssymb}
\usepackage{amsthm}
\usepackage{pst-blur}
\usepackage{pstricks-add}
\usepackage{pstricks}
\usepackage{pstricks-pdf}
\usepackage{pgf,tikz,pgfplots}
\usepackage{mathrsfs}
\usepackage{xcolor}
\usepackage{eso-pic}
%\usepackage{background}
\usepackage{xeCJK}
    \xeCJKsetup{CJKmath=true}
%\usepackage{wallpaper}

%opening
\title{}
\author{}
\date{}
\setlength\parindent{0em}

\begin{document}
\maketitle
\section{模拟题}{
设$n \in \mathbf{N^*},f(n)$为所有满足下列条件的整数数列$\{ a_0,a_1,…,a_n\}$的个数:\\
(1)$a_0=0,a_n=2n$,且$1\leq a_{k+1}=a_{k}\leq 3\quad(k=0,1,…,n-1)$;\\
(2)不存在$i,j \quad(0\leq i<j\leq <n)$,使得$a_j-a_i=n$.\\
求$3f(16)-2f(15)+f(10)$的值.
}
\iffalse

构造2×n的数表计数证明.
\fi 
\end{document}