\documentclass[]{article}


%导入宏包
\usepackage{ctex}
\usepackage{graphics}
\usepackage{amsmath,amssymb}
\usepackage{amsthm}
\usepackage{pst-blur}
\usepackage{pstricks-add}
\usepackage{pstricks}
\usepackage{pstricks-pdf}
\usepackage{pgf,tikz,pgfplots}
\usepackage{mathrsfs}
\usepackage{xcolor}
\usepackage{eso-pic}
%\usepackage{background}
\usepackage{xeCJK}
    \xeCJKsetup{CJKmath=true}
%\usepackage{wallpaper}

%opening
\title{}
\author{}
\date{}
\setlength\parindent{0em}


\begin{document}
\maketitle
\section{等力点性质应用(自编题)}{
在锐角$ \triangle EFG $中,点$ P $具有性质p,如果$ PA \times BC =PB \times AC =PC \times AB $\p 点$ P $具有性质q,如果$ PA+PB+PC $的值最小\p 点$ D $具有性质p,点$ K $具有性质q,$ L $,$ M $,$ N $为点K在$ EF $,$ EG $,$ GF $上的射影,点$ O $是$ \triangle LMN $中具有性质q的一点,$ S $为$ DK $中点,$ W为$  $\triangle LMN $的垂心.证明:$ SW \parallel KO $.
%\begin{proof}{}
%\end{proof}
}
\end{document}