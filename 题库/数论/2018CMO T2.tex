\documentclass[]{article}


%导入宏包
\usepackage{ctex}
\usepackage{graphics}
\usepackage{amsmath,amssymb}
\usepackage{amsthm}
\usepackage{pst-blur}
\usepackage{pstricks-add}
\usepackage{pstricks}
\usepackage{pstricks-pdf}
\usepackage{pgf,tikz,pgfplots}
\usepackage{mathrsfs}
\usepackage{xcolor}
\usepackage{eso-pic}
%\usepackage{background}
\usepackage{xeCJK}
    \xeCJKsetup{CJKmath=true}
%\usepackage{wallpaper}

%opening
\title{}
\author{}
\date{}
\setlength\parindent{0em}

\begin{document}
\maketitle
\section{2018CMO T2}{
若一个三元整数集合由某个直角三角形的三边长构成,则称其为“勾股三元集”.例如,$\{ 6,8,10 \}$为一个勾股三元集.证明:存在整数$m\geq 2$,及$m$个勾股三元集.设$P_1,P_2,…,P_m$满足$P_1=P,P_m=Q$,且对$1\leq i \leq m-1$,均有$P_i\cap P_{i+1}\ne \oslash$.
%\begin{proof}{}
%\end{proof}
}
\end{document}